%--------------------------------------------------------%
%	PREAMBLE
%--------------------------------------------------------%

% DOCUMENT CLASS
    \documentclass[letterpaper,12pt]{article}

%--------------------------------------------------------%  
% TITLE SECTION

 %Abstract
    \usepackage{abstract} % Allows abstract customization
    % Set the "Abstract" text to bold
    \renewcommand{\abstractnamefont}{\normalfont\bfseries}
    % Set the abstract itself to small italic text
    \renewcommand{\abstracttextfont}{\normalfont\small\itshape} 

 %Title
    \usepackage{titlesec} % Allows customization of titles

 %Authors
    \usepackage{authblk} % For multiple authors

 %Date
	\usepackage{datetime} % allows for including today's date
  	% These two lines creates a new date format ``Month day(th), year''
    \newdateformat{usvardate}{
  	\monthname[\THEMONTH] \ordinal{DAY}, \THEYEAR}
%--------------------------------------------------------%

%--------------------------------------------------------%
% HEADERS & FOOTERS

 %Footnotes
  	\usepackage[bottom]{footmisc} % Makes footnotes stick to bottom of the page
    
 %Headers from page 2 on
    \usepackage{fancyhdr}
    \pagestyle{fancy}
    \fancyheadoffset{0cm}
    \setlength{\headheight}{20pt} 
%--------------------------------------------------------%

% MACROS
    \providecommand{\keywords}[1]{\textbf{\textit{Keywords---}} #1}

% COMMANDS
    \newcommand{\etal}{{\em et al.\ }}

%% Updated how the wordcount is implemented
\newcommand\wordcount{%
  \immediate\write18{texcount -utf8 -merge -sum -incbib -dir -sub=none -brief \jobname.tex | cut -d : -f 1 > 'count.txt'}%
  \input{count.txt}\ignorespaces words%
}
  

% MATH SUPPORT
    % The amssymb package provides various useful mathematical symbols
    \usepackage{amssymb}
    % The amsthm package provides extended theorem environments
    \usepackage{amsmath}
    \newcommand{\floor}[1]{\lfloor #1 \rfloor}
    % The newtxmath package provides additional math symbol support
    	% in Times New Roman symbols, etc.
    \usepackage{newtxmath}

    
% Indention
    %This package allow using indent in the text
    \usepackage{changepage}
    % indention block will indent from the secound line of the pragraph
    \newenvironment{indentblock}{\begin{adjustwidth}{\parindent}{}\hspace{-\parindent}}{\end{adjustwidth}}
    \newenvironment{breakline}{\  \\}
    

% FONTS
    \usepackage{microtype} % Slightly tweak font spacing for aesthetics
    \usepackage[utf8]{inputenc}
    \usepackage{newtxtext} % Makes default font Adobe Times New Roman
    \usepackage[nottoc]{tocbibind}
    \usepackage[T1]{fontenc}
    \usepackage[square]{natbib}

  
% LINES
	% Spacing
	\usepackage{setspace} % See \doublespacing command at the top of content.tex
	\onehalfspacing
    % Numbering
    \usepackage{lineno,xcolor} 	% See \linenumbers at the top of content.tex

% MARGINS
	%NOTE: All spaces in this template are in inches, because it is
    % formatted for letterpaper (8.5 x 11 inch) paper. If you use a4
    % paper, choose different sizes in millimeters or centimeters.
	\usepackage[top=2in, bottom=2in, left=1.8in, right=1.8in]{geometry}

% COMMENTS
	\usepackage[colorinlistoftodos]{todonotes} % allows margin comments
    % See examples in content.tex, and here for manual: 
    % http://www.ctan.org/pkg/todonotes
	\usepackage{soul} % allows for highlighting
    
% GRAPHICS
    \usepackage{graphicx} % More advanced figure inclusion
    \usepackage{float} % For specifying table/figure locations, i.e. [ht!]
    
    % The printlen command allows the user to print the exact text width or height.
    % This is useful, when trying to create graphics (outside of LaTeX, of course)
    % with the optimal dimensions. See here for usage: http://www.ctan.org/pkg/printlen
    \usepackage{printlen}

% TABLES
    \usepackage{longtable} % For long tables that span multiple pages
    \newcommand{\sym}[1]{\rlap{#1}}% For symbols like *** in tables
    \usepackage{tabularx} % Allows advanced table features
    \newcolumntype{L}[1]{>{\raggedright\arraybackslash}p{#1}}
    \newcolumntype{C}[1]{>{\centering\arraybackslash}p{#1}}
    \newcolumntype{R}[1]{>{\raggedleft\arraybackslash}p{#1}}
    \usepackage{relsize} % Allows precise adjustment of font size,
    	%useful for fitting tables to page width

% REFERENCES
	\usepackage{hyperref} % For hyperlinks in the PDF
    \bibliographystyle{plain}


% algorithms
    %\usepackage{algorithm}% http://ctan.org/pkg/algorithm
    \usepackage{algpseudocode}% http://ctan.org/pkg/algorithmicx
    \usepackage[ruled,vlined,commentsnumbered,titlenotnumbered]{algorithm2e}

\usepackage{verbatim}
\usepackage{latexsym}
    
% Proofs

    \usepackage{amsthm}
    
    \newtheorem{theorem}{Theorem}
    \newtheorem{lemma}[theorem]{Lemma}
    \newtheorem{corollary}[theorem]{Corollary}
    \newtheorem{claim}[theorem]{Claim}
    \newtheorem{conjecture}[theorem]{Conjecture}
    \newtheorem{fact}[theorem]{Fact}
    \newtheorem{definition}[theorem]{Definition}
    \newtheorem{property}[theorem]{Property}


    
% Languages    
%    \usepackage[english,hebrew]{babel}
